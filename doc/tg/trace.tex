%%%%%%%%%%%%%%%%%%%%%%%%%%%%%%%%%%%%%%%%%%%%%%%%%%%%%%%%%%%%%%%%%%%%%%%%%%%%%%%%
% $Id: trace.tex 333 2015-06-30 13:00:39Z klugeflo $
%%%%%%%%%%%%%%%%%%%%%%%%%%%%%%%%%%%%%%%%%%%%%%%%%%%%%%%%%%%%%%%%%%%%%%%%%%%%%%%%

\section{Trace Generation}
\label{s:tracegen}

%%%%%%%%%%%%%%%%%%%%%%%%%%%%%%%%%%%%%%%%%%%%%%%%%%%%%%%%%%%%%%%%%%%%%%%%%%%%%%%%

%%%%%%%%%%%%%%%%%%%%%%%%%%%%%%%%%%%%%%%%%%%%%%%%%%%%%%%%%%%%%%%%%%%%%%%%%%%%%%%%

\subsection{Preliminary Considerations}
\label{ss:tracegen:prelim}

%%%%%%%%%%%%%%%%%%%%%%%%%%%%%%%%%%%%%%%%%%%%%%%%%%%%%%%%%%%%%%%%%%%%%%%%%%%%%%%%

Actual generation of traces on embedded platform should be as simple
as possible.
If enough memory is available, it would be possible to calculate all
signal times offline, such that the actual generator program simply runs
through a table.
Due to the vast memory requirements of such an approach, we utilise a
hybrid approach instead.
At least parts of the necessary calculations are performed offline.
Our aim is to implement trace generation through a composition of the
automatons described in sections~\ref{ss:model:rotary} and
\ref{ss:model:crank}.
Input data for these models is provided in pairs
$(t_N,\alpha_N)$, where $t_N$ is the time when the angular
acceleration changes to $\alpha_N$.

%%%%%%%%%%%%%%%%%%%%%%%%%%%%%%%%%%%%%%%%%%%%%%%%%%%%%%%%%%%%%%%%%%%%%%%%%%%%%%%%

\subsection{Preprocessor}
\label{ss:tracegen:preproc}

%%%%%%%%%%%%%%%%%%%%%%%%%%%%%%%%%%%%%%%%%%%%%%%%%%%%%%%%%%%%%%%%%%%%%%%%%%%%%%%%

The preprocessor transforms a driving cycle table (for examples, see
appendix~\ref{s:dc}) into an angular acceleration table for the actual
trace generator.
Thereby, the preprocessor proceeds as described in
section~\ref{ss:model:dotomega}.

%%%%%%%%%%%%%%%%%%%%%%%%%%%%%%%%%%%%%%%%%%%%%%%%%%%%%%%%%%%%%%%%%%%%%%%%%%%%%%%%

\subsection{Generator}
\label{ss:tracegen:generator}

%%%%%%%%%%%%%%%%%%%%%%%%%%%%%%%%%%%%%%%%%%%%%%%%%%%%%%%%%%%%%%%%%%%%%%%%%%%%%%%%

Tasks:
\begin{itemize}
\item set angular acceleration $\alpha$
\item develop $\varphi(t)$ and $\omega(t)$
\item generate signals $O_P, O_S$
\end{itemize}

Approach: Perform all relevant work in ISR for primary teeth.
Simplify implementation by setting a new $\alpha$ only when $O_P$
occurs.
$O_S$ timer is set in a corresponding (find an apt one) $O_P$
instance.
In the following, we use the expression \emph{phase} to denote the
time span between two changes of $\alpha$.
During a phase, $\alpha$ is constant.

%%%%%%%%%%%%%%%%%%%%%%%%%%%%%%%%%%%%%%%%%%%%%%%%%%%%%%%%%%%%%%%%%%%%%%%%%%%%%%%%

\subsubsection{Primary Teeth}

\begin{table}
  \caption{Parameters for primary teeth calculations}
  \label{t:parameters:primary}
  \vspace{1ex}
  \centering
  \begin{tabularx}{\textwidth}{c@{$\;:\ \;$}l@{\quad}X}
    \hline\hline
    $\Delta_P$ & $\mathbb{R}^+$ & Angular distance between two teeth\\
    $\varphi(t)$ & $\mathbb{R}^+_0$ & Angular position of crank shaft\\
    $\varphi^P_0$ & $\mathbb{R}^+_0$ & Initial position offset of
    crank shaft at beginning of current phase\\
    $\omega(t)$ & $\mathbb{R}^+_0$ & Angular velocity of crank shaft\\
    $\omega_0$ & $\mathbb{R}^+_0$ & Initial angular velocity of crank
    shaft at beginning of current phase\\
    $\alpha$ & $\mathbb{R} $ & Angular acceleration of crank
    shaft\\
    $t_0$ & $\mathbb{R}^+_0$ & Time offset of current phase\\
    \hline\hline
  \end{tabularx}
\end{table}

The parameters used in the following calculations are explained in
table~\ref{t:parameters:primary}.
The occurence of a primary tooth is characterised by the following
equation:
\begin{eqnarray}
  \varphi(t) - \varphi^P_0 \equiv 0 & (\mbox{mod } \Delta_P)
\end{eqnarray}
Through constraining a change of $\alpha$ to times where $O_P$
occurs, we can simplify this equation and search for the $k_P$-th
primary signal in a phase that is characterised by:
\begin{equation}
  \label{eq:trace:p:kth}
  \varphi(t) = \varphi^P_0 + k_P\Delta_P
\end{equation}

Solve equation~\eqref{eq:trace:p:kth}:
\begin{eqnarray}
  & \frac{1}{2}\alpha(t-t_0)^2+\omega_0(t-t_0)+\varphi^P_0 =
  \varphi^P_0 + k\Delta_P&\\
  \Leftrightarrow & \frac{1}{2}\alpha t^2 + \omega_0t -
  \frac{1}{2}\alpha t_0^2-\omega_0t_0 - k\Delta_P=0 &\\
  \Leftrightarrow & t_{1/2} = \frac{-\omega_0\pm\sqrt{\omega_0^2 +
      2\alpha\left(\frac{1}{2}\alpha t_0^2+\omega_0t_0 + k\Delta_P\right)}}{\alpha}&
\end{eqnarray}

We are only interested in the right-hand solution, as $t(k)$ must be a
monotonically increasing function:
\begin{equation}
  t(k) = \frac{-\omega_0 + \sqrt{\omega_0^2 +
      2\alpha\left(\frac{1}{2}\alpha t_0^2 +
      \omega_0t_0\right) + 2\alpha k\Delta_P}}{\alpha}
\end{equation}
Parts of the discriminant must only be calculated when a new
phase begins.
Defining two constants
\begin{eqnarray}
  D_1 & := & \omega_0^2 +
  2\alpha\left(\frac{1}{2}\alpha t_0^2 +
  \omega_0t_0\right)\\
  D_2 & := & 2\alpha\Delta_P
\end{eqnarray}
we can simplify $t(k)$:
\begin{equation}
  \label{eq:tk}
    t(k) = \frac{-\omega_0 + \sqrt{D_1 + D_2k}}{\alpha}
\end{equation}

The above calculations apply mainly to the primary signal $O_P$.
To generate $O_P$ concurrently, a closer look is necessary.
The basic approach is to perform all calculations only in the IRQ
handler of the $O_P$ routine.
Concerning $O_S$, a phase shift parameter
$\phi_S\in[0,\Delta_P)$ is introduced that describes the
angular distance betweeen the secondary tooth and the preceding
primary tooth.

%%%%%%%%%%%%%%%%%%%%%%%%%%%%%%%%%%%%%%%%%%%%%%%%%%%%%%%%%%%%%%%%%%%%%%%%%%%%%%%%

\subsubsection{Secondary Teeth}

\begin{table}
  \caption{Additional parameters for secondary teeth calculations}
  \label{t:parameters:secondary}
  \vspace{1ex}
  \centering
  \begin{tabularx}{\textwidth}{c@{$\;:\ \;$}l@{\quad}X}
    \hline\hline
    $\varphi^S_0$ & $\mathbb{R}^+_0$ & Initial position offset of next tooth\\
    $\Delta_S$ & $\mathbb{R}^+$ & Angular distance between two secondary teeth\\
    $\phi_S$ & $\mathbb{R}^+_0$ & Angular distance between next
    secondary tooth and primary tooth at which last phase change was triggered\\
    \hline\hline
  \end{tabularx}
\end{table}

Calculations of the secondary teeth times require additional
parameters that are summarised in table~\ref{t:parameters:secondary}.
Times for $O_S$ can be developed in a similar manner, but the
calculations will be a bit more complex, as a change of $\alpha$
will not necessarily overlap with an occurrence of $O_S$.
Thus, $O_S$ will occur any time when:
\begin{eqnarray}
  \varphi(t) - \varphi^S_0 \equiv 0 & (\mbox{mod } \Delta_S)
\end{eqnarray}
Ideally, the time of $O_S$ should just be calculated when the preceeding
primary tooth was triggered, to account for a possible change of
$\alpha$.

The $k$-th secondary signal in a single phase has the following
property:
\begin{equation}
  \label{eq:trace:s:kth}
  \varphi(t) = \varphi_0 + \phi_S + (k-1)\Delta_S
\end{equation}


%%%%%%%%%%%%%%%%%%%%%%%%%%%%%%%%%%%%%%%%%%%%%%%%%%%%%%%%%%%%%%%%%%%%%%%%%%%%%%%%

\subsubsection{Implementation Notes}

\paragraph{Time Units}
Use seconds.

\paragraph{Replace Multiplication by Addition}
Implementation should use equations~\eqref{eq:trace:p:kth} and
\eqref{eq:trace:s:kth}:
The $k$ parameter will only implicitly be kept, instead implementation
should simply advance the $k\Delta_P$ resp. $(k-1)\Delta_S$ products.

\paragraph{Keep Numbers small}
In the long rung, the angular position $\varphi(t)$ can grow to a very
large number.
Reset $\varphi(t)$ in regular intervals, e.g once per revolution, to
avoid loosing precision.

\paragraph{Irrelevance of Phase Shift}
For the use case, the initial phasing of the crank shaft is
irrelevant.
Therefore, set $\varphi_0:=0$.

\paragraph{Renormalisation}
Renormalisation is performed at $\varphi(t)=0\;(\mbox{mod }1)$.
\begin{eqnarray}
  \omega_0 & := & \omega(t)\\
  \varphi_0 & := & 0\\
  t_0 & := & t\\
  \leftrightarrow t & := & 0\\
\end{eqnarray}

\paragraph{Secondary Tooth}
\label{p:trace:sectooth}
Let the secondary tooth reside directly after the third primary tooth
at $\phi_S\in\left[3\Delta_P,4\Delta_P\right)$.

\paragraph{Phase Change}
May happen at any primary tooth.
Some phases are very short (1-2 teeth).
Restricting phase changes to only a single tooth would make the
discriminant in calculation of next $t$ (eq.~\eqref{eq:tk}) negative.
 Additional work:
\begin{equation}
  \alpha := \alpha_N
\end{equation}

\paragraph{Setting Secondary Tooth Timer}
Performed in ISR for second primary tooth.
Depending on the actual $\phi_S$, even the third ISR might suffice,
but we want to be on the safe side.

%%%%%%%%%%%%%%%%%%%%%%%%%%%%%%%%%%%%%%%%%%%%%%%%%%%%%%%%%%%%%%%%%%%%%%%%%%%%%%%%
%%% Local Variables: 
%%% mode: latex
%%% TeX-master: tg
%%% TeX-PDF-mode: t
%%% End: 
%%%%%%%%%%%%%%%%%%%%%%%%%%%%%%%%%%%%%%%%%%%%%%%%%%%%%%%%%%%%%%%%%%%%%%%%%%%%%%%%
%<!-- Local IspellDict: english -->
